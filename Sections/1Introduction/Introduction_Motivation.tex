%Two-particle azimuthal correlations in different kinematic ranges have played a crucial role in the understanding of jet production
%and modification in relativistic heavy-ion collisions, from RHIC to LHC energies.
%The extension of these studies to the heavy-flavour domain at the LHC will probe our understanding of QCD in the perturbative
%regime, accessible in a large kinematic range given the large mass of heavy quarks. Flavour conservation in QCD implies that charm quarks5are always produced as pairs of quarks and anti-quarks. The azimuthal correlations obtained using a meson carrying a heavy quark as trigger particle
%with the other charged particles in the same event give the possibility to study the underlying charm production mechanism in detail. In particular, prompt
%heavy quark pair production leads, at first order in leading-order pQCD, to back-to-back production. Heavy quarks produced by splitting of massless gluons are instead
% produced at small $\Delta\phi$. A third production process, the production
%by flavour excitation leads to a big separation in rapidity and most of the time one of the two quarks is produced in the forward region. \\

%Angular differential measurements of charm production are therefore interesting in their own right in pp collisions to study the structure
%and fragmentation of charm quarks, as well as their production mechanism. Correlations of  prompt charm meson pairs were first measured by Fermilab E791 Collaboration at Tevatron \cite{E791} also later by
%CDF II at Tevatron, showing a fair agreement with Pythia for the overall production, with some disagreement in the collinear and back-to-back production~\cite{CDF}. \\

%Heavy-flavour correlation studies in more complex collisional systems, like Pb-Pb aim to study the modification of the fragmentation of charmed jets due to in-medium (or cold nuclear matter, in case of p-Pb collisions) effects, in a similar fashion as it was done for di-hadron correlation studies in heavy-ion collisions (see for example~\cite{th1,th2}). Furthermore, the recent observation of long range correlations in p-Pb for light flavour hadrons and for heavy-flavour decay electrons %add reference
%points to possible collective effects or effects originating from gluon saturation in the initial state. More information could be extracted by the eventual observation of the same effect with D mesons.\\

%In the following, we describe the analysis strategy for p-Pb in all its steps, and we describe the list of corrections and the estimation of the systematic uncertainties
%done. After a section dedicated to detailed studies performed with Monte Carlo simulations to check the different analysis steps, we present the results of
%$\Delta\phi$ correlations obtained for prompt $D^0$, $D^{+}$ and $D^{*+}$ in different ranges of transverse momentum for the D-meson (trigger particle) and the associated particles.

The study of the azimuthal correlations of heavy-flavour particles and charged particles at the LHC energies provides a way to characterize charm production and fragmentation processes in pp collisions. The measurement also provide a way to probe our understanding of QCD in the perturbative regime, accessible in a large kinematic range given the large mass of heavy quarks. Flavour conservation in QCD implies that charm quarks are always produced as pairs of quarks and anti-quarks. The azimuthal correlations obtained using a meson carrying a heavy quark as trigger particle with the other charged particles in the same event give the possibility to study the underlying charm production mechanism in detail. In particular, prompt charm quark-antiquark pair production is back to back in azimuth at first order in leading-order perturbative-QCD (pQCD). If a hadron from the quark hadronization is taken as trigger particle, a near-side (at $\Delta\varphi = 0$) and an away-side (at $\Delta\varphi = \pi$) peaks would appear in the azimuthal correlation distributions, coming from the fragmentation of the quark pair. Heavy quarks produced from the splitting of a massless gluon can be rather collimated and may generate sprays of hadrons at small $\Delta\varphi$. Finally, for hard-scattering topologies classified as ``flavour-excitation", a charm quark undergoes a hard interaction from an initial splitting ($g\to c\bar{c}$), leading to a big separation in rapidity of the hadrons originating from the antiquark (quark) with respect to the trigger D meson and contribute to a rather flat term to the $\Delta\varphi$-correlation distribution.

%Heavy-flavour correlation studies in more complex collision systems, like Pb-Pb, play a crucial role in studying the modification of the fragmentation of charmed jets due to in-medium (or cold nuclear matter, in case of p-Pb collisions) effects, in a similar way as it was done for di-hadron correlation studies in heavy-ion collisions (see for example \cite{ALICEPbPbdih}). Furthermore, the recent observation of long range correlations in p-Pb for light flavour hadrons (\cite{ALICEv2ppb}, \cite{ALICEv2ppb2}) and for heavy-flavour decay electrons (\cite{ALICEv2HFe}) points to possible collective effects or effects originating from gluon saturation in the initial state. More information could be extracted by the eventual observation of the same effect with D mesons.\\

In the following note, we first describe the analysis strategy for the pp 2017 data sample in all its steps, followed by the list of analysis corrections and the estimation of systematic uncertainties. Finally the results of $\Delta\varphi$ correlations, and quantitative observable extracted to fits to those distributions, obtained for prompt $\rm{D^0}$, $\rm{D^{+}}$ and $\rm{D^{*+}}$ in different ranges of transverse momentum for the D-meson (trigger particle) and the associated particles are presented.

The extension of the momentum ranges (both for D mesons and associated particles) with respect to the 2016 pp dataset, as well as the improved precision in the common ranges allow a more thorough investigation of the charm quark fragmentation properties (multiplicity of tracks as a function of momentum, geometrical profile of charm jets, $\pt$ distribution of the tracks inside the jet). This can also allow us to put better constraints of charm fragmentation and charm jet properties provided by models.
Furthermore, 2017 pp data sample allows us a direct and more reasonable comparison with 2016 p-Pb data, since it has the same center-of-mass energy and, thanks to the higher precision and statistics, it was possible to exploit the azimuthal correlations in the same (extended and more differential) pT ranges of both the trigger and the associated particle of the 2016 p-Pb data sample.
This new pp reference data, together with new p-Pb 2016 data will help to study cold nuclear matter effects  affecting the charm fragmentation in p-Pb with better precision.
In addition, this new pp data can also be used as solid and precise references in view of an analysis on a Pb-Pb sample at the same energy.
