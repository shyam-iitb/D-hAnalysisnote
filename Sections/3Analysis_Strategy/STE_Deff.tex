{\bf \normalsize (i) Tracking efficiency} - The tracking efficiency was calculated by obtaining the ratio between the yield at the reconstructed level and generated level, for a defined ``type" of particles (in our case non-identified particles) and it is estimated differentially in p$_T$, $\eta$, and z$_{vtx}$ of the charged particles.\\

Tracking efficiency maps were produced as TH3D histograms (p$_T$, $\eta$, z$_{vtx}$) obtained from MC analysis on the minimum-bias samples LHC17f2b$\_$fast and LHC17f2b$\_$cent$\_$woSDD, considering only primary pions, kaons, protons, electrons and muons, and applying at reconstructed level the track selections (summarized in Table.~\ref{table:effCuts}). These efficiency maps were used in the analysis tasks to extract single track efficiencies; each correlation pairs found in the data analysis was inserted in correlation plots with a weight of {\bf 1/efficiency value}. 
As a cross-check, the tracking efficiency was evaluated, with the same criteria, also on the LHC17f2a$\_$fast and LHC17f2a$\_$cent$\_$woSDD samples, which were produced with EPOS-LHC generator instead of DPMJET. Compatibility within 1\% between the efficiency values on the two samples was found.
The 1D ($\pt$ dependence) tracking efficiency, evaluated on f2b samples (blue) and on f2a samples (red) are shown in Fig.~\ref{fig:trackeff}, as well as the ratio of f2b over f2a efficiencies.

\begin{figure}[h]
	\centering
	\includegraphics[width=0.8\linewidth]{figures/Effs/CompareEff_FiveSpecies.png}
    \includegraphics[width=0.8\linewidth]{figures/Effs/RatioEff_FiveSpecies.png}
	\caption{1D (vs $\pt$) tracking efficiency map for standard track selection, evaluated on f2b samples (blue) and f2a samples (red) on top panel, and their ratios on bottom panel.}
	\label{fig:trackeff}	
\end{figure}

\newpage
Details of cuts at event level and particle/track selection at different steps are listed in Table ~\ref{table:effCuts} . \\
\begin{table}[h]
\small
\centering % used for centering table

\begin{tabular}{  p{5cm} |  p{8.5cm} }
 \\
  \multirow{1}{*}{\large \textbf {MC Generated }} \\
\hline
\\
     Stages         &              Cuts \\
\hline\hline & \\		            	
  1.MC Part with Generated Cuts         &    {\textbf {After Event Selection}}\\
																		   & Charge\\
																		    & PDG Code\\
														  				  & Physical Primary \\

   2. MC Part with Kine Cuts         &              {\textbf {Kinematics Cuts }}\\
															    & -0.8$\textless \eta \textless  0.8$\\
															    & $\pt$ $\textgreater$ 0.3 (GeV/$c$)\\

&		\\            	


\multirow{1}{*}{\large \textbf {MC Reconstructed }} & \\
\hline


\hline & \\		            	                        	
4. Reco tracks        &                             {\textbf {After Event Selection}}\\
															   & Physical Primary \\
															
															
5. Reco tracks with Kine Cuts         &               {\textbf  {Kinematics Cuts }}\\
															    & -0.8$\textless \eta \textless  0.8$\\
															    & $\pt$ $\textgreater$ 0.3 (GeV/$c$)\\



6. MC true with Quality Cuts         &      			      {\textbf  {Quality Cuts }} \\
																	&SetRequireSigmaToVertex(kFALSE) \\
																	&SetDCAToVertex2D(kFALSE) \\
																	&SetMinNCrossedRowsTPC(70)\\
																	&SetMinRatioCrossedRowsOverFindableClustersTPC(0.8)\\
																	&SetMinNClustersITS(2)\\
																	&SetMaxChi2PerClusterTPC(4)\\
																	&SetMaxDCAToVertexZ(1) \\
																	&SetMaxDCAToVertexXY(1) \\
																	&SetRequireTPCRefit(TRUE) \\
																	&SetRequireITSRefit(FALSE) \\

7. Reco tracks with Quality Cuts         &             {\textbf  {Same as step 6}} \\

 &\\		            	            		

 \hline \hline
 \\
\end{tabular}
\caption{\large {The list of event and particle/track selection cuts used in the estimation of single track efficiency}} % title of Table
\label{table:effCuts}	
\end{table}

{\bf \large (ii) D meson efficiency} - Due to limited statistics, the correlation analysis is performed in quite wide $\pt$ bins and in each of them the reconstruction and selection efficiency of D mesons is not flat, in particular in the lower $\pt$ region. We correct for the $\pt$ dependence of the trigger efficiency within each p$_\mathrm{T}$-bin.

This correction is applied online, by using a map of D meson efficiency as a function of $\pt$ and event multiplicity (in terms of SPD tracklets in $|\eta|<1$) extracted from the enriched Monte Carlo sample LHC17d2a$\_$fast$\_$new. The $\eta$ dependence was neglected due to the statistics of the available Monte Carlo sample, which rule out the possibility of performing a 3D study.

To properly count the number of trigger particles used to normalize the correlation distributions, $N_\text{trig}$, each D meson is weighted with the inverse of its efficiency
in the invariant mass distribution. The main role of the correction for the D meson efficiency is to account for the $\pt$ dependence of the correlation distribution within a given D meson $\pt$ interval. Indeed, only the $\pt$ shape of the D meson efficiency within the correlation $\pt^{\rm trig}$ ranges is relevant while the average value
in the $\pt$ range is simplified due to the normalization of the correlation distribution to the number of trigger particles.

Efficiency plots for $\Dzero$, $\Dplus$ and $\Dstar$ mesons are shown in Figs.~\ref{fig:dEffPrompt} and ~\ref{fig:dEffFD}.

\begin{figure}[!htp]     %da c
	\centering
%Marianna
    \includegraphics[width=.48\linewidth]{figures/Effs/EfficiencyMap_2D_DPlus_c_Ref_wLimAcc_Plot.png}
	\includegraphics[width=.48\linewidth]{figures/Effs/EfficiencyMap_1D_DPlus_c_Ref_wLimAcc_Plot.png}  % by Fabio
	\includegraphics[width=.48\linewidth]{figures/Effs/EfficiencyMap_2D_DStar_c_Ref_wLimAcc_Plot.png}
	\includegraphics[width=.48\linewidth]{figures/Effs/EfficiencyMap_1D_DStar_c_Ref_wLimAcc_Plot.png}  % by Fabio
	\includegraphics[width=.48\linewidth]{figures/Effs/EfficiencyMap_2D_Dzero_c_RefPtBins_wLimAcc_Plot.png}
	\includegraphics[width=.48\linewidth]{figures/Effs/EfficiencyMap_1D_Dzero_c_RefPtBins_wLimAcc_Plot.png}  % by Fabio
	
	%\includegraphics[width=.30\linewidth]{figures/D0Eff_ProjMult_3to4GeV.png}
	%\includegraphics[width=.30\linewidth]{figures/D0Eff_ProjMult_5to6GeV.png}
	%\includegraphics[width=.30\linewidth]{figures/D0Eff_ProjMult_8to12GeV.png}
\caption{Top panel: ($\pt$, multiplicity) dependence (left) and $\pt$  dependence (right) of prompt $\Dplus$ meson efficiency.
Mid panel: ($\pt$, multiplicity) dependence (left) and $\pt$  dependence (right) of prompt $\Dstar$ meson efficiency.
Bottom panel: ($\pt$, multiplicity) dependence (left) and $\pt$  dependence (right) of prompt $\Dzero$ meson efficiency.
%s: multiplicity dependence of $D^0$ meson efficiency for three $D^0$ p$_\mathrm{T}$ ranges: 3-4 GeV/$c$ (left), 5-6 GeV/$c$ (center), 8-12 GeV/$c$ (right). For tracklet multiplicity$>$ 120, due to the limited statistics, the efficiency value is fixed to the one obtained for 90$<$tracklet multiplicity$<$120.
}
	\label{fig:dEffPrompt}	
\end{figure}

\begin{figure}[h]   %da B
	\centering
	%Marianna
		  % by Fabio
	\includegraphics[width=.48\linewidth]{figures/Effs/EfficiencyMap_2D_DPlus_b_Ref_wLimAcc_Plot.png}
	\includegraphics[width=.48\linewidth]{figures/Effs/EfficiencyMap_1D_DPlus_b_Ref_wLimAcc_Plot.png}
	  % by Fabio
	\includegraphics[width=.48\linewidth]{figures/Effs/EfficiencyMap_2D_DStar_b_Ref_wLimAcc_Plot.png}
	\includegraphics[width=.48\linewidth]{figures/Effs/EfficiencyMap_1D_DStar_b_Ref_wLimAcc_Plot.png}
	  % by Fabio
	\includegraphics[width=.48\linewidth]{figures/Effs/EfficiencyMap_2D_Dzero_b_RefPtBins_wLimAcc_Plot.png}
	\includegraphics[width=.48\linewidth]{figures/Effs/EfficiencyMap_1D_Dzero_b_RefPtBins_wLimAcc_Plot.png}
	\caption{Top panel: ($\pt$, multiplicity) dependence (left) and $\pt$ dependence (right) of feed-down $\Dplus$ meson efficiency.
Mid panel: ($\pt$, multiplicity) dependence (left) and $\pt$ dependence (right) of feed-down $\Dstar$ meson efficiency.
Bottom panel: ($\pt$, multiplicity) dependence (left) and $\pt$ dependence (right) of feed-down $\Dzero$ meson efficiency.}
	\label{fig:dEffFD}	
\end{figure}
\clearpage

It was observed that multiplicity dependence of the efficiency does not bias the extraction of the signal yield from the invariant mass distributions (which, as anticipated, are also weighted in the same manner). In addition, the multiplicity dependence of the efficiencies (shown for the $\Dzero$, in integrated $\pt$ range, in Fig. \ref{fig:DeffY}) is rather flat in the range 20-80 tracklets, where about 90\% of the reconstructed D-mesons are found, which explains why it has a negligible effect on the correlation distributions on this data sample.
\begin{figure}[h]   %da B
	\centering
	\includegraphics[width=.48\linewidth]{figures/Effs/EfficiencyMap_1D_Dzero_c_RefPtBins_Ydep_wLimAcc_Plot.png}
\caption{Prompt $\Dzero$ meson efficiency as a function of multiplicity (SPD tracklet in $|\eta|<1$).}
	\label{fig:DeffY}	
\end{figure}
\clearpage 
